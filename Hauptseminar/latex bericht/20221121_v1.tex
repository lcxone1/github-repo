\documentclass[lettersize,journal]{IEEEtran}
\usepackage{amsmath,amsfonts}
\usepackage{algorithmic}
\usepackage{array}
\usepackage[caption=false,font=normalsize,labelfont=sf,textfont=sf]{subfig}
\usepackage{textcomp}
\usepackage{stfloats}
\usepackage{url}
\usepackage{verbatim}
\usepackage{graphicx}
\hyphenation{op-tical net-works semi-conduc-tor IEEE-Xplore}
\def\BibTeX{{\rm B\kern-.05em{\sc i\kern-.025em b}\kern-.08em
    T\kern-.1667em\lower.7ex\hbox{E}\kern-.125emX}}
\usepackage{balance}
\begin{document}
\title{Tools for Energy demand forecasting: state of the art world wide}
\author{Chengxi Li}
% \thanks{Manuscript created October, 2020; This work was developed by the IEEE Publication Technology Department. This work is distributed under the \LaTeX \ Project Public License (LPPL) ( http://www.latex-project.org/ ) version 1.3. A copy of the LPPL, version 1.3, is included in the base \LaTeX \ documentation of all distributions of \LaTeX \ released 2003/12/01 or later. The opinions expressed here are entirely that of the author. No warranty is expressed or implied. User assumes all risk.}}

% \markboth{Journal of \LaTeX\ Class Files,~Vol.~18, No.~9, September~2020}%
% {How to Use the IEEEtran \LaTeX \ Templates}

\maketitle

\begin{abstract}
  Energy demand forecasting is a salient process for proper allocation of the available resources for industrial production, agricultural, health, population and education etc. It is also the starting point for building roadmaps for achiving a low carbnon energy systems. In the last decades, researchers have contributed thousands of papers on forecasting of future energy demand.
  In this article I made an attemptation to briefly review the various energy demand forecasting methodes world wide since 2005. XX valuable open data sources were also been pointed out. No artificial Intelligence method are discussed in this article.
  
\end{abstract}

\begin{IEEEkeywords}
Energy demand forecasting, Energy models, Forecasting model
\end{IEEEkeywords}


\section{Introduction}

Projecting long-term energy demand at the global aggregate level is the starting point for creating a comprehensive roadmap for the transition to an achieve a climate neutral world by mid-century \cite{UNFCCC}. While energy forecasting can be interpreted as forecasting of kWh consumption, I use the wider definition, referring to forecasting of the energy industry. We pay particular attention to topics related to energy systems, including electricity demand, and wind and solar power generation. Although oil and gas forecasting is also an important subset of energy forecasting, it is outside the scope of this article.

[Mehr zu Introduction]

The rest of this article is structured as follows: Section 2 presents an overview of the literature; Section 3 introduces xx valuable data sources and emphasizes the importance of reproducible research; Section 4 summarizes this article.

\section{Overview of Literatures}

  \subsection{Time Series Models}
Time series models uses time series trend analysis for extrapolating the future energy requirement. Himanshu and Lester \cite{srilanka} have used time series analysis for predicting electricity demand in Sri Lanka. Mabel et al. used pearl or logistic function to forecast the future wind energy patterns in India\cite{TSA-india}.
  
  \subsection{Probabilistic Forecasting}
  Many phenomena and systems in nature can be viewed as or modeled by stochastic processes. The frequently used point forecasts, or single-valued forecasts, are simply presenting summary statistics, mostly expected values, of a subject during different time periods. In weather forecasting, it has long been known that a forecast is essentially fivedimensional, spanning the three-dimensional space, time and probability. [mehr info] 

  Some specific examples can be found in the area of energy demand forecasting. Researchers have proposed various forecast combination strategies to generate and improve probabilistic load forecasts \cite{37}, \cite{76}, \cite{77}. In \cite{37}, the researchers [genauer erzaehlen was hat der Autor gemacht]. In \cite{76}, [genauer erzaehlen was hat der Autor gemacht]. In \cite{77}, [genauer erzaehlen was hat der Autor gemacht]

  \subsection{Hierarchical Forecasting}
  
    \subsubsection{Bottom-Up Approach}
  Energy forecasting often encounters time series that have aggregation constraints due to temporal or geographical groupings. In these scenarios, hierarchical forecasting, which reconciles base forecasts generated individually at different levels of a hierarchy, becomes important. There are two ways of hierarchical forecasting, bottom-up and top-down. The 'bottom-up' modelling approach are based on a detailed analysis of individual behaviour, disaggregated in continents, countries, regions, and main economic sectors—industry, transport, buildings, agriculture, etc. The criteria used to disaggregate may vary, but generally they tend to be quite detailed, reaching the smallest possible decision units and from several points of view. [hier vor- und nachteile]  In \cite{9}, [erzaehlen was hat der Autor gemacht]

    \subsubsection{Top-Down Approach}
  An alternative approach, by contrast, starts from the global perspective and disaggregates the analysis as long as there are data and well-established behavioural parameterisations available. This could be characterised as a 'top-down' approach, as opposed to the 'bottom-up' disaggregated approach discussed previously.In contrast to bottom-up approach, the number of paper using this model for long-term forecasting is much smaller. One reason could be the lack of a solid microfoundation, implying that the aggregated relationships, reasonable as they may be, are only loosely backed by theory. Unexpected breaks may derail forecasts, and that is what should be carefully scrutinised when assessing long-range forecasts and simulations.

  
  \subsection{Grey Prediction Models}
  nergy demand forecasting can be regarded as grey system problem, because a few factors such as GDP, income, population are known to influence the energy demand but how exactly they affect the energy demand is not clear.

  Energy consumption in China is forecast using grey prediction model which incorporates genetic algorithm\cite{272}.[genauer erzaehlen was hat der Autor gemacht]

  \subsection{Regression Models?}








\section{Open Data Sources}
  \subsection{Data published with ties to papers}
  \subsection{ISO Data}
  \subsection{In situ Weather Data}
  \subsection{Economics and population Data}

\section{Summary}

\begin{thebibliography}{1}
  
  \bibitem{UNFCCC}
  UNFCCC. “The Paris Agreement.” United Nations Framework Convention on Climate Change, United Nations, 2016, unfccc.int/process-and-meetings/the-paris-agreement/the-paris-agreement.
  
  \bibitem{srilanka}
  Himanshu AA, Lester CH. Electricity demand for Sri Lanka: a time series analysis. Energy 2008;33:724–39.

  \bibitem{TSA-india}
  Mabel MC, Fernandez E. Growth and future trends of wind energy in India. Renewable and Sustainable Energy Reviews 2008;12:1745–57.

  \bibitem{37}
  B. Liu, J. Nowotarski, T. Hong, and R. Weron, ‘‘Probabilistic load forecasting via quantile regression averaging on sister forecasts,’’ IEEETrans. Smart Grid, vol. 8, no. 2, pp. 730–737, Mar. 2017.

  \bibitem{76}
  Y. Wang, N. Zhang, Y. Tan, T. Hong, D. S. Kirschen, and C. Kang, ‘‘Combining probabilistic load forecasts,’’ IEEE Trans. Smart Grid, vol. 10, no. 4, pp. 3664–3674, Jul. 2019

  \bibitem{77}
  T. Li, Y. Wang, and N. Zhang, ‘‘Combining probability density forecasts for power electrical loads,’’ IEEE Trans. Smart Grid, vol. 11, no. 2, pp. 1679–1690, Mar. 2020.

  \bibitem{9}
  Boßmann, T.; Staffell, I. The shape of future electricity demand: Exploring load curves in 2050s Germany and Britain. Energy 2015, 90, 1317–1333

  \bibitem{272}
  Lee Y-S, Tong L-I. Forecasting energy consumption using a grey model improved by incorporating genetic programming. Energy Conversion and Management 2011;52:147–52.

\end{thebibliography}

\end{document}


