\documentclass[lettersize,journal]{IEEEtran}
\usepackage{amsmath,amsfonts}
\usepackage{algorithmic}
\usepackage{array}
\usepackage[caption=false,font=normalsize,labelfont=sf,textfont=sf]{subfig}
\usepackage{textcomp}
\usepackage{stfloats}
\usepackage{url}
\usepackage{verbatim}
\usepackage{graphicx}
\hyphenation{op-tical net-works semi-conduc-tor IEEE-Xplore}
\def\BibTeX{{\rm B\kern-.05em{\sc i\kern-.025em b}\kern-.08em
    T\kern-.1667em\lower.7ex\hbox{E}\kern-.125emX}}
\usepackage{balance}
\begin{document}
\title{Tools for Energy demand forecasting: state of the art world wide}
\author{Chengxi Li}
% \thanks{Manuscript created October, 2020; This work was developed by the IEEE Publication Technology Department. This work is distributed under the \LaTeX \ Project Public License (LPPL) ( http://www.latex-project.org/ ) version 1.3. A copy of the LPPL, version 1.3, is included in the base \LaTeX \ documentation of all distributions of \LaTeX \ released 2003/12/01 or later. The opinions expressed here are entirely that of the author. No warranty is expressed or implied. User assumes all risk.}}

% \markboth{Journal of \LaTeX\ Class Files,~Vol.~18, No.~9, September~2020}%
% {How to Use the IEEEtran \LaTeX \ Templates}

\maketitle

\begin{abstract}
  Energy demand forecasting is a crucial process for the proper allocation of available resources for various sectors, such as industrial production, agriculture, health, population, and education. It involves using models and analysis to predict future energy demand and evaluate the impacts of different policy scenarios on energy markets, the environment, and economic performance. Governments, utilities, and other organizations rely on energy demand forecasting as a tool to plan for the future and ensure that energy systems can meet the needs of society. In the last decades, researchers have contributed thousands of papers on forecasting of future energy demand. In this article I made an attemptation to briefly review the various energy demand forecasting methodes world wide since 2005. Among them, I have detailed the following 4 models: TIMES, LEAP, PRIME, and AIM/ENDUSE. No artificial Intelligence method are discussed in this article.
  
\end{abstract}

\begin{IEEEkeywords}
Energy demand forecasting, Energy models, Forecasting model
\end{IEEEkeywords}


\section{Introduction}

Energy demand forecasting is a crucial process for the proper allocation of available resources for various sectors, such as industrial production, agriculture, health, population, and education. It involves using models and analysis to predict future energy demand and evaluate the impacts of different policy scenarios on energy markets, the environment, and economic performance. Governments, utilities, and other organizations rely on energy demand forecasting as a tool to plan for the future and ensure that energy systems can meet the needs of society. This process is an integral part of energy planning and policy development, allowing decision-makers to anticipate future energy requirements and identify potential challenges and opportunities. Additionally, energy demand forecasting plays a key role in the transition to more sustainable and resilient energy systems, helping to identify the most effective and efficient pathways for meeting future energy demand in an environmentally and sustainably responsible manner.

Projecting long-term energy demand at the global aggregate level is the starting point for creating a comprehensive roadmap for the transition to an achieve a climate neutral world by mid-century \cite{UNFCCC}. While energy forecasting can be interpreted as forecasting of kWh consumption, I use the wider definition, referring to forecasting of the energy industry. We pay particular attention to topics related to energy systems, including electricity demand, and wind and solar power generation. Although oil and gas forecasting is also an important subset of energy forecasting, it is outside the scope of this article.

[wer hat ansonst was schon gemacht]

The goal of the report, is to provide an briefly overview of the various energy demand forecasting methodes world wide. Meanwhile I provide an in-depth understanding of the models using the following four models as an example: TIMES, LEAP, PRIME, and AIM/ENDUSE.

The remainder of this article is organized as follows: In Section 2, I provide an overview of modeling techniques used in energy demand forecasting. In Section 3, I provide an in-depth explanation of each of these four models, including their original authors, the methodology employed, the energy sector they focus on, and the time horizon in which they can be applied. Finally, in Section 4, I summarize the key points of this article.
\section{Overview of Literatures}

  \subsection{Time Series Models}

Recording the ordered sequence of the values of a variable at fixed time intervals creates a time series. Time series models are the most simplest of models which uses time series trend analysis for extrapolating the future energy requirement. Time series models are often categorized as top-down models, and represent the relationship between the variable's values with time.  

Himanshu and Lester \cite{srilanka} have used time series analysis for predicting electricity demand in Sri Lanka. Mabel et al. used pearl or logistic function to forecast the future wind energy patterns in India \cite{TSA-india}.

Three time series models, namely, Grey-Markov model, Grey-Model with rolling mechanism, and singular spectrum analysis (SSA) are used to forecast the consumption of conventional energy in India. GreyMarkov model has been employed to forecast crude-petroleum consumption while Grey-Model with rolling mechanism to forecast coal, electricity (in utilities) consumption and SSA to predict natural gas consumption \cite{time-serie india}.

The problem of forecasting the monthly peak demand of electricity in north India was studied by Ghosh \cite{time-serie india ghosh}. In this paper the authors used two different time series methods–that of multiplicative seasonal autoregressive integrated moving average and Holt-Winters multiplicative exponential smoothing. 

Mati et al. \cite{time-serie Nigeria}used the time series to forecast the electricity demand in Nigeria. A multiple regression time series was applied, and electricity consumption and percentage connectivity to the national grid were considered the independent variables of the model

Moreover, a chaotic time series method was used by Wang et al. \cite{time-serie china}to forecast the electricity demand. The authors believed that the electricity demand series had the chaotic characteristics, and that the
proposed method could effectively predict the demand with a mean absolute relative error of 2.48\%. It is notable that the seasonal effects were considered in the method by using a trend adjustment technique; additionally, a data set from the network of New South Wales in Australia was used to simulate the needed data.

Simmhan and Noor \cite{time series clustering} applied the incremental clustering of time series to forecast the energy consumption. The main contribution of the authors is about applying the method to big data and used 700,000 input data points to show the efficacy of the developed model both in terms of accuracy and prediction time. 

  \subsection{Probabilistic Forecasting}
  Many phenomena and systems in nature can be viewed as or modeled by stochastic processes. The frequently used point forecasts, or single-valued forecasts, are simply presenting summary statistics, mostly expected values, of a subject during different time periods. In weather forecasting, it has long been known that a forecast is essentially fivedimensional, spanning the three-dimensional space, time and probability. [mehr info] 

  Some specific examples can be found in the area of energy demand forecasting. Researchers have proposed various forecast combination strategies to generate and improve probabilistic load forecasts \cite{37}, \cite{76}, \cite{77}. In \cite{37}, the researchers [genauer erzaehlen was hat der Autor gemacht]. In \cite{76}, [genauer erzaehlen was hat der Autor gemacht]. In \cite{77}, [genauer erzaehlen was hat der Autor gemacht]

  \subsection{Hierarchical Forecasting}
  
    \subsubsection{Bottom-Up Approach}
  Energy forecasting often encounters time series that have aggregation constraints due to temporal or geographical groupings. In these scenarios, hierarchical forecasting, which reconciles base forecasts generated individually at different levels of a hierarchy, becomes important. There are two ways of hierarchical forecasting, bottom-up and top-down. The 'bottom-up' modelling approach are based on a detailed analysis of individual behaviour, disaggregated in continents, countries, regions, and main economic sectors—industry, transport, buildings, agriculture, etc. The criteria used to disaggregate may vary, but generally they tend to be quite detailed, reaching the smallest possible decision units and from several points of view. [hier vor- und nachteile]  In \cite{9}, [erzaehlen was hat der Autor gemacht]

    \subsubsection{Top-Down Approach}
  An alternative approach, by contrast, starts from the global perspective and disaggregates the analysis as long as there are data and well-established behavioural parameterisations available. This could be characterised as a 'top-down' approach, as opposed to the 'bottom-up' disaggregated approach discussed previously.In contrast to bottom-up approach, the number of paper using this model for long-term forecasting is much smaller. One reason could be the lack of a solid microfoundation, implying that the aggregated relationships, reasonable as they may be, are only loosely backed by theory. Unexpected breaks may derail forecasts, and that is what should be carefully scrutinised when assessing long-range forecasts and simulations.

  
  \subsection{Grey Prediction Models}
  nergy demand forecasting can be regarded as grey system problem, because a few factors such as GDP, income, population are known to influence the energy demand but how exactly they affect the energy demand is not clear.

  Energy consumption in China is forecast using grey prediction model which incorporates genetic algorithm\cite{272}.[genauer erzaehlen was hat der Autor gemacht]

  \subsection{Regression Models?}








\section{Open Data Sources}
  \subsection{Data published with ties to papers}
  \subsection{ISO Data}
  \subsection{In situ Weather Data}
  \subsection{Economics and population Data}

\section{Summary}

\begin{thebibliography}{1}
  
  \bibitem{UNFCCC}
  UNFCCC. “The Paris Agreement.” United Nations Framework Convention on Climate Change, United Nations, 2016, unfccc.int/process-and-meetings/the-paris-agreement/the-paris-agreement.
  
  \bibitem{srilanka}
  Himanshu AA, Lester CH. Electricity demand for Sri Lanka: a time series analysis. Energy 2008;33:724–39.

  \bibitem{TSA-india}
  Mabel MC, Fernandez E. Growth and future trends of wind energy in India. Renewable and Sustainable Energy Reviews 2008;12:1745–57.

  \bibitem{37}
  B. Liu, J. Nowotarski, T. Hong, and R. Weron, ‘‘Probabilistic load forecasting via quantile regression averaging on sister forecasts,’’ IEEETrans. Smart Grid, vol. 8, no. 2, pp. 730–737, Mar. 2017.

  \bibitem{76}
  Y. Wang, N. Zhang, Y. Tan, T. Hong, D. S. Kirschen, and C. Kang, ‘‘Combining probabilistic load forecasts,’’ IEEE Trans. Smart Grid, vol. 10, no. 4, pp. 3664–3674, Jul. 2019

  \bibitem{77}
  T. Li, Y. Wang, and N. Zhang, ‘‘Combining probability density forecasts for power electrical loads,’’ IEEE Trans. Smart Grid, vol. 11, no. 2, pp. 1679–1690, Mar. 2020.

  \bibitem{9}
  Boßmann, T.; Staffell, I. The shape of future electricity demand: Exploring load curves in 2050s Germany and Britain. Energy 2015, 90, 1317–1333

  \bibitem{272}
  Lee Y-S, Tong L-I. Forecasting energy consumption using a grey model improved by incorporating genetic programming. Energy Conversion and Management 2011;52:147–52.

  \bibitem{time-serie india}
  Kumar U, Jain VK. Time series models (Grey-Markov. Grey Model with rolling mechanism and singular spectrum analysis) to forecast energy consumption in India. Energy 2010; 35(4): 1709–16

  \bibitem{time-serie india ghosh}
  Ghosh, S.: Univariate time-series forecasting of monthly peak demand of electricity in northern India. Int. J. Indian Cult. Bus. Manag. 1(4), 466–474 (2008)

  \bibitem{time-serie Nigeria}
  Mati, A.A., Gajoga, B.G., Jimoh, B., Adegobye, A., Dajab, D.D.: Electricity demand forecasting in Nigeria using time series model. Pac. J. Sci. Technol. 10(2), 479–485 (2009)

  \bibitem{time-serie china}
  Wang, J., Chi, D., Wu, J., Lu, H.Y.: Chaotic time series method combined with particle swarm optimization and trend adjustment for electricity demand forecasting. Expert Syst. Appl. 38(7), 8419–8429 (2011)

  \bibitem{time series clustering}
  Simmhan, Y., Noor, M.U.: Scalable prediction of energy consumption using incremental time series clustering. In: Big Data, IEEE International Conference, pp. 29–36. IEEE (2013)
\end{thebibliography}

\end{document}


